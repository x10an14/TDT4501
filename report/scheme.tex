% !TEX root = ./report.tex

\clearpage
\section{Inlining Heuristics using an RVSDG}

\todo[inline]{\textbf{\textit{Scheme section}}}

The project described in this report implements the heuristic(s) deciding which
functions to inline, in the Jive compiler using an RVSDG IR. This section
discusses first the utilization of the Heuristics used, then continues with how
their implementation in this project.

\subsection{Application of the Heuristics}

The heuristics used in the project described by this report can be divided into
to main subcategories: the heuristics used to decide the order of which
\textit{apply}-nodes to look at first, and the heuristics used to decide whether
or not to inline a single \textit{apply}-node.

When traversing the RVSDG representing the program being compiled, the order of
which \textit{apply}-nodes we find first makes no difference to the approach
used in this project. However, the ordering used to sort the
\textit{apply}-nodes \textit{found} makes a difference, due to the ordering
deciding which \textit{apply}-node might be inlined first.

The
heuristic used
to order the \textit{apply}-node
in this project uses the relative
position of each \textit{apply}-node.
The further away from the root-node (which
represents the final operation of the program),
the higher the value representing this relative position.

\todo[inline]{Draw figure showing why this makes a difference.}

Figure~\ref{X} illustrates why the order of functions inlined make a difference.
Had we not traversed the RVSDG collecting all the \applyNode s before inlining
them, the the impact of the inlining order would not be as simple to measure.

\todo{What else can I say about this?}Using this approach, we can test the
consequences of different heuristical orderings of \applyNode s in the RVSDG
worked on by the compiler.

\subsection{Implementation of the Heuristics}

The codebase of the project this report describes, is implemented in C/C++. As
such, \textit{Standard Template Library} (STL) additions such as vectors,
templates, and containers, in addition to the \textit{Object Oriented} (OO)
features in C++, are used in the implementation of the inliner.

\subsubsection{Traversing the graph for \textit{apply}-nodes}

First, we traverse the graph given as a parameter to the function we have
analogous to \lstinline[style=customcpp]!main()!, namely
\lstinline[style=customcpp]!apply_inliner_heuristic()!.

\subsubsection{Ordering the \textit{apply}-nodes of the graph}

\subsubsection{Whether to inline each \textit{apply}-nodes}
