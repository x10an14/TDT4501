% !TEX root = ./report.tex

\clearpage
\section{Inlining Heuristics using an RVSDG}

The project described in this report implements the heuristic(s) deciding which
functions to inline, in the Jive compiler using an RVSDG IR.

\subsection{Application of the Heuristics}

The heuristics used in the project described by this report can be divided into
to main subcategories: the heuristics of which \textit{apply}-nodes to look at
first, and the heuristics of whether or not to inline a single
\textit{apply}-node.

When traversing the RVSDG representing the program being compiled, the order of
which \textit{apply}-nodes we find first makes no difference to the approach
used in this project. However, the ordering used to sort the \textit{found}
\textit{apply}-nodes makes a difference, due to this order deciding which
\textit{apply}-node might be inlined first.

The heuristic used to order the \textit{apply}-node in this project uses the
relative position of each \textit{apply}-node. The further away from the
root-node which represents the final operation of the program, the higher the
value representing this relative position.

\todo[inline]{Draw figure showing why this makes a difference.}

\subsection{Implementation of the Heuristics}

\subsubsection{Traversing the graph for \textit{apply}-nodes}

\subsubsection{Ordering the \textit{apply}-nodes of the graph}

\subsubsection{Whether to inline each \textit{apply}-nodes}
