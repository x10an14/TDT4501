\subsection{List of unread papers}

\begin{itemize}

	\item ``\textbf{Should potential loop optimizations influence inlining
decisions?}'', by Christopher Barton and José Nelson Amaral and Bob Blainey, at
University of Alberta and IBM Toronto Software Laboratory, Canada. \\
\textit{Perhaps 2nd most interesting at this moment?}\\
\url{http://dl.acm.org/citation.cfm?id=961329}

	\item ``\textbf{Region-based compilation: an introduction and motivation}'',
\\ byRichard E. Hank and Wen-Mei W. Hwu, at University of Illinois, and B.
Ramakrishna Rau, at Hewlett Packard Laboratories, Palo Alto California. \\
\textit{3rd most interesting on this list at the moment.} \\
\url{http://dl.acm.org/citation.cfm?id=225160.225189&coll=DL&dl=ACM&CFID=470941312&CFTOKEN=25125832}

	\item ``\textbf{Statically Unrolling Recursion to Improve Opportunities for
Parallelism}'', \\by Neil Deshpande and Stephen A. Edwards, at Columbia
University NY. \\
\textit{This one sounds cool, and being static is a plus. But that's all its got
going for it without having read it more closely. It does discuss inlining.}\\
\url{http://www.cs.columbia.edu/~sedwards/papers/deshpande2012statically.pdf}

	\item ``\textbf{An adaptive strategy for inline substitution}'', \\ by Keith
D. Cooper and Timothy J. Harvey and Todd Waterman, at Rice University Houston
and Texas Instruments Inc. Stafford, Texas. \\
\textit{Perhaps most interesting at this time due to discussing global/static
optimizations as well as being relatively new compared to other papers in this
list?}\\
\url{http://dl.acm.org/citation.cfm?id=1788381}

	\item ``\textbf{Adaptive compilation and inlining}'', \\ by Todd Waterman
and Keith D. Cooper, at Rice University Houston Texas. \\
\textit{Not sure if this is what Nico wants me to find, but this is a paper
discussing aspects of inlining I don't think I've considered much in earlier
paper searches.}\\
\url{http://dl.acm.org/citation.cfm?id=1195126}

	\item ``\textbf{Practical and effective higher-order optimizations}'', \\ by
Lars Bergstrom and Matthew Fluet and Matthew Le and John Reppy and Nora Sandler,
at Mozilla Research CA and Rochester Institute of Technology NY and University
of Chicago IL. \\
\textit{This paper discusses inlining as well as some other optimizations.
Picked for citing GHC-paper, and being from 2014, more than much else.
I'm not sure I completely understand the abstract, but I got the feeling the
paper might have an interesting take on inlining.} \\
\url{http://dl.acm.org/citation.cfm?id=2628153}

	\item ``\textbf{Demand-driven Inlining Heuristics in a Region-based
Optimizing Compiler for ILP Architectures*}'', \\ by Tom Way and Ben Breech and
Wei Du and Lori Pollock, at University of Delaware USA. \\
\textit{This paper discusses a region-based compiling technique and how to
utilize inlining in conjunction with this, while focusing on optimizations for
ILP architectures. Picked paper for mentioning inlining in Keywords and
abstract, as well as also dealing with region based compilation which seems
relatable to Jive.} \\
\url{http://www.eecis.udel.edu/~hiper/passages/papers/waypdcsiasted01.pdf}

\end{itemize}

\subsection{Read papers}

Below papers are listed according to perceived relevance/usefulness for this
project.\\

\begin{enumerate}

	\item ``\textbf{Adaptive Compilation and Inlining}'', by Todd
Waterman (Ph.D. Thesis)

Waterman's Ph.D. thesis \cite{AdaptvCompilAndInlingWaterman} examines the use of
techniques to adaptively decide which functions to inline during program
compilation, to show that they can be used to improve the performance of
specific optimizations.

The adaptive technique both matches, and at times, beats ATLAS \todo{need ATLAS
cite?} and gcc on one of the object oriented biggest programs in the SPEC
CINT2000 test suite. The technique accepts condition strings (like in
\cite{AutoTuningJavaHeuristics}) to determine which call sites are inlined. The
condition strings provide a flexible inliner by combining various program
properties in CNF, and exposes a large space of different inlining decisions
with the potential to outperform static techniques when used adaptively.

	\item ``\textbf{Secrets of the Glasgow Haskell Compiler inliner}'',
\cite{GHC-paper}

This paper explains how inlining is dealt with by the GHC compiler. It spreads
this into the following sections:
\begin{enumerate}
	\item Short section describing what inlining is, and its most obvious and
generic properties and pitfalls (Section 2).
	\item An explanation of a strategy of how (and when) to inline recursive
functions (Section 3).
	\item How they so deal with name capture (Section 4).
	\item As well as discussing certain moments/situations when inlining is
carefully considered by the GHC. They also have a section describe how they
exploit their name-capture solution to support accurate tracking of both lexical
and evalutation-state environments (Section 5+6).
	\item Finally they sketch their implementation (Section 7).
\end{enumerate}

	\item ``\textbf{Inline expansion: when and how?}'',
\cite{InlineWhenHowSerrano}

This paper focuses on \textit{when} and \textit{how} to inline functions,
classified as either \textit{recursive} or \textit{non-recursive} functions.
This is done in the programming language Scheme, with \todo{confirm this?}the
Bigloo compiler M. Serrano has created.

The paper gives an algorithm for when to inline (which \cite{GHC-paper} remarks
upon), as well as comments and results on their inlining wrt. the two classes of
functions the paper defines.

	\item ``\textbf{Automatic Tuning of Inlining Heuristics}'',
\cite{AutoTuningJavaHeuristics}

This paper explores and explains the advantages of genetic algorithm  heuristics
when dynamically compiling Java. Since it is dynamically compiled, it's harder
for the compilation unit to get a proper global view of the compilation job,
required resources, and beneficial trade-offs. All of which play important roles
when deciding whether to incur the cost of increased code size when inlining.

	\item ``\textbf{Should Portential Loop Optimizations Influence Inlining Decisions?}'',

This paper discusses (as the title suggests) whether inlining should be improved
upon or neglected, so as to improve potential loop optimizations. The authors of
this paper tests their hypothesis on the Toronto Portable Optimizer (TPO), and
their results show that loop-fusion is not affected in any major scale by
furthering or removing inlining decisions. Their results are all comparable with
the IBM XL compiler suite as is.

They do however, discuss some of IBM XL compiler suite's criteria for inlining,
but only very briefly in section 3.3, and I've not been able to find how their
``Correctness'' criteria is found/decided. The paper mentions no technical aspects of how they decide whether or not to inline in the rest of the paper.

	\item ``\textbf{Region-Based Compilation: An Introduction and Motivation}'',

Instruction level parallelism (\textbf{ILP}) has become an increasing focus in
further development and improvement in code efficiency. This paper looks at how
in a new technique, region-based compilation, can use inlining of functions to
further the achievement of better code efficiency. The authors of this paper
used the \textit{Multiflow} compiler when doing their research for this paper.

	\item ``\textbf{Statitcally Unrolling Recursion to Improve Opportunities for Parallelism}'',

This paper focuses on how inlining should be done for the GHC, and how it is
useful, and proving the correctness of their technique. It does not discuss when
or why you should inline certain functions to any useful degree, beyond the
usual ``recursive functions can be very useful to inline with this technique''.

\end{enumerate}
