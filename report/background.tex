% !TEX root = ./report.tex

\section{Background}

In this section we first explain why inlining is a practice found in almost
every compiler to date, before we list (in order of relevance)\todo{To do} the
papers contributing to the background of this report, and finally summarize what
each paper brings to this report.

\subsection{Inlining}
\todo[inline]{Decide whether to keep or throw out.\\
	If throw out, perhaps re-arrange structure wrt. Related Work.}

\subsection{Related Work}


\begin{itemize}
	\item ``\textbf{Secrets of the Glasgow Haskell Compiler inliner}'', written
by Simon Peyton Jones and Simon Marlow, at Microsoft. \\
	\todo[inline]{Summary of paper.}

	\item ``\textbf{Automatic Autotuning of Inlining Heuristics}'', written by
John Cavazos and Michael F.P. O'Boyle, at School of Informatics University of
Edinburg. \\
	\todo[inline]{Summary of paper.}

\end{itemize}
