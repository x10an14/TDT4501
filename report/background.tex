% !TEX root = ./report.tex

\section{Background}

\subsection{The Regionalized Value-State Dependency Graph}
\label{background:RVSDG}

\begin{itemize}

	\item \textit{$\gamma$} blocks/regions in the RVSDG are conditionals. The
inputs of each $\gamma$ block is the variables(/data) upon which the conditional
depends, as well as the operation performed on these. A CNF may then be
represented as several nested $\gamma$ blocks, all the while retaining the
properties of a demand-dependence graph.

	\item \textit{$\theta$} blocks/regions are the loop constructs.
\todo[inline]{Need better explanation.}

	\item \textit{$\lambda$} blocks/regions are the functions. They retain the
demand-dependency graph properties by implementing\todo{Ask Nico for help?}

	\item \textit{$\phi$} blocks/regions are the ones describing mutually
recursive environments. Inside a $\phi$ region, the RVSDG will have at least one
$\lambda$ region, and perhaps more. If there is more than one $\lambda$ present,
then the ``mutually recursive binding group'' situation which P. Jones and
Marlow \cite{GHC-paper} describe when discussing how to inline recursive
functions, is also present in the program flow represented by this RVSDG.

\end{itemize}
