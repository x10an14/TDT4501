% !TEX root = ./report.tex

\section{Background}
\label{background}

\subsection{The Regionalized Value-State Dependence Graph}
\label{background:RVSDG}

The RVSDG is a directed acyclic graph (DAG). The RVSDG has different types of
nodes, and two types of edges. The nodes are the $\gamma$-, $\theta$-,
$\lambda$-, apply-, and $\phi$-nodes. The edges will be discussed first.

\todo[inline]{Finish this introduction of the graph, don't forget operations,
``simple nodes'', like +, - and so on. \\ ``Once a node contains a subgraph,
it's a complex node''.}

There are two types of edges in the RVSDG.

The first type is the data-dependence edge. This edge describes a data
dependence one node has from its parent node.

The other type of edge is the state-dependence edge. This edge is meant to keep
the ordering of the nodes consistent with the original flow of execution.
Stipled lines are used to denote state-dependence edges.

See figure \todo{Make Fibonacci figure example with RVSDG nodes with a printf in
main()}X for an example of why state-dependence edges are necessary for the
RVSDG.

\subsubsection{N-way statements}

\textit{$\gamma$}-nodes represent conditional statements. Each
$\gamma$-node has two sets of inputs: the predicate, and the data dependencies
the predicate depend upon. The outputs are any data dependencies used further in
the graph (program).

If-else statements are represented as a $\gamma$-node, which is split vertically
in two. Each subsection has the subgraph and operations to be performed if the
predicate is evaluated to true and false, respectively.

\subsubsection{Tail-controlled loops}

\textit{$\theta$}-nodes represent tail loops in the program. They are equivalent
to do-while loops containing the representation of the body of the loop, as well
as an edge onwards out of the nodes onto the next node in the graph.

A for-loop would be presented with an if-else $\gamma$-node, with a
$\theta$-node in the body of the $\gamma$-node representing the ``true''
subgraph.

\subsubsection{Functions}

\textit{$\lambda$}-nodes represent functions, and \textit{apply}-nodes are
``call-nodes'' which represent where a function is called in the RVSDG.

\subsubsection{Mutually recursive functions}

\textit{$\phi$}-regions are nodes representing parts of the program's
control flow where either a functions behave recursively either by calling
themselves (mutually recursive), or two or more calling each other in turn.

A $\phi$-region needs to have at least one apply node and one $\gamma$-node. A
mutually recursive function will have an output edge from the $\gamma$-node
going back to the start of the apply node supplying the input for said
$\gamma$-node.

\todo[inline]{Abovementioned Fibonacci example with examples of graphnodes,
maybe code too?}
