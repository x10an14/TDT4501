% !TEX root = ./report.tex

\section{Background}
\label{background}

\subsection{The Regionalized Value-State Dependency Graph}
\label{background:RVSDG}

The RVSDG is a directed acyclic graph (DAG). The RVSDGhas different types of
nodes, and two types of edges. The nodes are the $\gamma$-, $\theta$-,
$\lambda$-, apply-, and $\phi$-nodes. The edges will be discussed first.

\subsubsection{Data-dependency edges}

There are two types of edges in the RVSDG.

The first type is the data-dependency edge. This edge describes a data-
dependency one noded has from its parent node. In other words, an edge saying
which nodes need to finish first before the necessary data is ready for the
current node.

\subsubsection{State-dependency edges}

The other type of edge is the state-dependency edge. This edge is meant to keep
the ordering of the nodes consistent with the program, when there are no data-
dependencies between them. Stipled lines are used to denote state-dependency
edges.

See figure \todo{Make figure}X for an example of why state-dependency edges are
necessary for the RVSDG.

\subsubsection{N-way statements}

\textit{$\gamma$}-nodes in the RVSDG represent conditional statements. Each
$\gamma$-node has two sets of inputs: the predicate, and the data dependencies
the predicate depend upon. The outputs are any data dependencies used further in
the graph (program).

If-else statements are represented as a $\gamma$-node, which is split vertically
in two. Each subsection has the subgraph and operations to be performed if the
predicate is evaluated to true and false, respectively.

\subsubsection{Tail-controlled loops}

\textit{$\theta$}-nodes represent loops in the program. They are do-while -loops
containing the representation of the body of the loop, as well as an edge
onwards out of the nodes onto the next node in the graph. $\theta$-nodes

A for-loop would be presented with an if-else $\gamma$-node, with a
$\theta$-node in the body of the $\gamma$-node representing the ``true''
subgraph

\subsubsection{Functions}

\textit{$\lambda$}-nodes represent functions, and \textit{Apply}-nodes are
``call-nodes'' which represent where a function is called in the RVSDG.

\subsubsection{Mutually recursive functions}

\textit{$\phi$}-regions are nodes representing parts of the program's
control flow where either a functions behave recursively either by calling
themselves (mutually recursive), or each two or more calling each other in turn.

A $\phi$-region needs to have at least one apply node and one $\gamma$-node. A
mutually recursive function will have an output edge from the $\gamma$-node
going back to the start of the apply node supplying the input for said $\gamma$-
node.
