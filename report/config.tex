% !TEX root = ./report.tex

\usepackage{float}			%For forcing position of figures
\usepackage{fancybox}		%For Sbox used for CenteredBox environment
\usepackage[usenames,%
dvipsnames]{xcolor}			%For the SkyBlue background color for lstlistings
\usepackage{tabularx}		%for tablecontents wrapping inside cell, instead of cell breaking page width.
\usepackage{enumerate}		%For getting different types of lists like a) II) and so forth.

\usepackage[T1]{fontenc}
\usepackage{textcomp}
\usepackage[utf8]{inputenc}	%For norwegian letters and UTF8 encoding support
\usepackage{lastpage}		%For the command \pageref{lastpage}
%\usepackage{pdfpages}		%For including pdfs
\usepackage{appendix}		%For appendix in article
\usepackage[nottoc,%
numbib]{tocbibind}			%For references in Table of Contents
\usepackage[english]{babel}	%For filltext in conjunction with blindtext
\usepackage{blindtext}		%For filltext (see comment for babel)
\usepackage[justification=centering]{caption}		%For use of the \caption command inside minipage environment
\usepackage{standalone}		%For being able to inlclude project_description which has begin{document}
\usepackage{geometry}		%For scaling/setting margin size and so on
%\usepackage{svg}			%For includesvg command
%\usepackage{textgreek}	%Test

%\setsvg{svgpath = figures/}
%\setsvg{inkscape = inkscape -z -D}

\usepackage{xargs}	% Use more than one optional parameter in a new commands
\usepackage[colorinlistoftodos,prependcaption]{todonotes} %For \todo
\newcommandx{\thiswillnotshow}[2][1=]{\todo[disable,#1]{#2}}
\newcommandx{\unsure}[2][1=]{\todo[linecolor=red,backgroundcolor=red!25,bordercolor=red,#1]{#2}}
\newcommandx{\change}[2][1=]{\todo[linecolor=blue,backgroundcolor=blue!25,bordercolor=blue,#1]{#2}}
\newcommandx{\info}[2][1=]{\todo[linecolor=OliveGreen,backgroundcolor=OliveGreen!25,bordercolor=OliveGreen,#1]{#2}}
\newcommandx{\improvement}[2][1=]{\todo[linecolor=Plum,backgroundcolor=Plum!25,bordercolor=Plum,#1]{#2}}

\usepackage[colorlinks]{hyperref}	%For \url{}
\hypersetup{linkcolor=black}
\hypersetup{citecolor=black}
\hypersetup{urlcolor=black}

\DeclareRobustCommand{\applyNode}{\textit{apply}-node}
\DeclareRobustCommand{\ApplyNode}{\textit{Apply}-node}

\usepackage{datetime}		%For \newdateformat
\newdateformat{CurMonth}{\monthname[\THEMONTH],~\THEYEAR}

\usepackage{listings}		%For code in document

\makeatletter
\newenvironment{CenteredBox}{%
\begin{Sbox}}{% Save the content in a box
\end{Sbox}\centerline{\parbox{\wd\@Sbox}{\TheSbox}}}% And output it centered
\makeatother

\lstset{
	tabsize=4,					% sets default tabsize to 4 spaces
	frame=none,					% adds a frame around the code
	numbers=left,				% where to put the line-numbers
	stepnumber=1,				% the step between two line-numbers. If it is 1 each line will be numbered
	numbersep=5pt,				% how far the line-numbers are from the code
	captionpos=b,				% sets the caption-position to bottom
	language=C++,				% choose the language of the code
	showtabs=false,				% show tabs within strings adding particular underscores
	breaklines=true,			% sets automatic line breaking
	keywordstyle={},			% sets the graphical style of keywords in the language
	showspaces=false,			% show spaces adding particular underscores
	numberstyle=\tiny,			% the size of the fonts that are used for the line-numbers
	escapeinside={@}{@},		% if you want to add a comment within your code
	showstringspaces=false,		% underline spaces within strings
	breakatwhitespace=false,	% sets if automatic breaks should only happen at whitespace
	basicstyle=\footnotesize,	% the size of the fonts that are used for the code
	backgroundcolor=\color{White},	% choose the background color. You must add \usepackage{color}
}

\lstdefinestyle{minipage_customcpp}{
	%frame=l,							%
	tabsize=2,							% sets default tabsize to 2 spaces
	language=C++,						%
	numbers=none,						% where to put the line-numbers
	stepnumber=1,						% the step between two line-numbers. If it is 1 each line will be numbered
	captionpos=b,						% sets the caption-position to bottom
	numbersep=5pt,						% how far the line-numbers are from the code
	showtabs=false,						% show tabs within strings adding particular underscores
	breaklines=true,					% sets automatic line breaking
	showspaces=false,					% show spaces adding particular underscores
	numberstyle=\tiny,					% the size of the fonts that are used for the line-numbers
	escapeinside={@}{@},				% if you want to add a comment within your code
	showstringspaces=false,				% underline spaces within string
	breakatwhitespace=true,				% sets if automatic breaks should only happen at whitespace
	basicstyle=\footnotesize,			%
	stringstyle=\color{Orange},			%
	identifierstyle=\color{Blue},		%
	backgroundcolor=\color{White},		% choose the background color. You must add \usepackage{color}
	belowcaptionskip=1\baselineskip,	%
	commentstyle=\itshape\color{Brown},	%
	keywordstyle=\bfseries\color{Green},% sets the graphical style of keywords in the language
}

\lstdefinestyle{global_customcpp}{
	%frame=l,							%
	tabsize=2,							% sets default tabsize to 2 spaces
	language=C++,						%
	numbers=none,						% where to put the line-numbers
	stepnumber=1,						% the step between two line-numbers. If it is 1 each line will be numbered
	captionpos=b,						% sets the caption-position to bottom
	numbersep=5pt,						% how far the line-numbers are from the code
	showtabs=false,						% show tabs within strings adding particular underscores
	breaklines=true,					% sets automatic line breaking
	showspaces=false,					% show spaces adding particular underscores
	numberstyle=\tiny,					% the size of the fonts that are used for the line-numbers
	escapeinside={@}{@},				% if you want to add a comment within your code
	showstringspaces=false,				% underline spaces within strings
	breakatwhitespace=true,				% sets if automatic breaks should only happen at whitespace
	basicstyle=\footnotesize,			%
	stringstyle=\color{Orange},			%
	identifierstyle=\color{Blue},		%
	backgroundcolor=\color{White},		% choose the background color. You must add \usepackage{color}
	belowcaptionskip=1\baselineskip,	%
	commentstyle=\itshape\color{Brown},	%
	keywordstyle=\bfseries\color{Green},% sets the graphical style of keywords in the language
}

\lstset{literate=
	{á}{{\'a}}1 {é}{{\'e}}1 {í}{{\'i}}1 {ó}{{\'o}}1 {ú}{{\'u}}1
	{Á}{{\'A}}1 {É}{{\'E}}1 {Í}{{\'I}}1 {Ó}{{\'O}}1 {Ú}{{\'U}}1
	{à}{{\`a}}1 {è}{{\'e}}1 {ì}{{\`i}}1 {ò}{{\`o}}1 {ù}{{\`u}}1
	{À}{{\`A}}1 {È}{{\'E}}1 {Ì}{{\`I}}1 {Ò}{{\`O}}1 {Ù}{{\`U}}1
	{ä}{{\"a}}1 {ë}{{\"e}}1 {ï}{{\"i}}1 {ö}{{\"o}}1 {ü}{{\"u}}1
	{Ä}{{\"A}}1 {Ë}{{\"E}}1 {Ï}{{\"I}}1 {Ö}{{\"O}}1 {Ü}{{\"U}}1
	{â}{{\^a}}1 {ê}{{\^e}}1 {î}{{\^i}}1 {ô}{{\^o}}1 {û}{{\^u}}1
	{Â}{{\^A}}1 {Ê}{{\^E}}1 {Î}{{\^I}}1 {Ô}{{\^O}}1 {Û}{{\^U}}1
	{œ}{{\oe}}1 {Œ}{{\OE}}1 {æ}{{\ae}}1 {Æ}{{\AE}}1 {ß}{{\ss}}1
	{ç}{{\c c}}1 {Ç}{{\c C}}1 {ø}{{\o}}1 {å}{{\r a}}1 {Å}{{\r A}}1
	{€}{{\EUR}}1 {£}{{\pounds}}1
}
