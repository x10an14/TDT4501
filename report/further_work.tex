% !TEX root = ./report.tex

\clearpage
\section{Further Work}
\label{sec:further_work}

In this section we've collected our ideas for potential future work discovered
while working on the inliner for this project. First, in
Section~\ref{sub:fw:adaptiv_comp} and Section~\ref{sub:fw:optimal_loop_breakers},
we will discuss potential future work already described in
literature~\cite{AdaptvStratInlSubst,AdaptvCompilAndInlingWaterman,BasMscThesis},
before finishing with more original ideas for future work in \todo{Add final
section reference if written.}Section~\ref{sub:fw:call_site_visit_order}.

\subsection{Dynamic Profiling, Parametrization, and Adaptive Compilation}
\label{sub:fw:adaptiv_comp}

As mentioned in Section~\ref{sub:scheme:inlining_apply_nodes} and
Section~\ref{sub:meth:SPEC2006_heuristics},
Waterman~\cite{AdaptvCompilAndInlingWaterman} implemented a hillclimber
algorithm to dynamically and adaptively re-compile a program so as to find the
optimal inlining heuristic for each compilation.

This is something we wanted to implement for the inliner of this project, but
did not have the resources for. It is our belief that the strengths of Jive's
RVSDG approach to compilation/inlining can be further exploited if the choosing
of inlining heuristic was done dynamically and adaptively like Waterman
proposes.

The work of Cooper, et al.~\cite{AdaptvStratInlSubst} shows how the
parametrization of the inliner Waterman~\cite{AdaptvCompilAndInlingWaterman}
proposed can easily be expanded upon and added to, making it a more powerful
tool. This is also something we believe that Jive could benefit from wrt. to
inlining, but did not have the resources to implement.

\subsection{Choosing loop breakers more carefully}
\label{sub:fw:optimal_loop_breakers}

When deciding upon recursive functions to be marked as \textit{loop breakers},
discussed in Section~\ref{sub:scheme:inlining_recur_apply_nodes}, our approach
does not choose these carefully. Ideally, an approach such as the one
Bas~\cite{BasMscThesis} proposes, would be more ideal because it makes an effort
in choosing the minimum amount of loop breakers needed.

Again, due to time and resources, we did not get the chance to implement Bas's
suggested approach based on the Directed Blackout Feedback Vertex Set problem.
However, it's utility in inlining functions belonging to recursive environments
is obvious, and its implementation in Jive's inliner would have been
advantageous.

\subsection{The ordering of which call sites to order first}
\label{sub:fw:call_site_visit_order}

To our knowledge, no other literature exist on the subject of the consequences
of the ordering of inlined functions. As illustrated with the example from
Section~\ref{sub:scheme:ordering_apply_nodes}, the order in which inlines are
performed affects not only the total count of inlines, but also \textit{which}
call sites are inlined.

Since the RVSDG gives us a directed call graph for each program compiled, other
approaches can be tested. One such approach could be to see whether there are
properties ignored/gained which are inherent to the ordering of the call sites
in a top-down ordering compared to a bottom-up ordering.

Another approach could be to find the aggregate cost of inlining sequential call
sites: profiling them first, before inlining them. Would this approach hide
worthwile potential optimizations found if the call sites in the chain were
inlined one by one? Or would more worthwile optimizations be found, if this
approach gives an increased chance of applying optimization techniques like
\textit{Common Subexpression Elimination}?
