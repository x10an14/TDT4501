% !TEX root = ./report.tex

\clearpage
\section{Further Work}
\label{sec:further_work}

In this section we've collected our ideas for potential future work discovered
while working on the inliner for this project. First, in
Section~\ref{sub:fw:adaptiv_comp} and
Section~\ref{sub:fw:optimal_loop_breakers}, discuss potential future work
already described in
literature~\cite{AdaptvStratInlSubst,AdaptvCompilAndInlingWaterman,BasMscThesis}, as well as quickly give our opinion of their applications
to this project/Jive. Thereafter we finish with a more original idea for future
work in Section~\ref{sub:fw:call_site_visit_order}.

\subsection{Dynamic Profiling, Parametrization, and Adaptive Compilation}
\label{sub:fw:adaptiv_comp}

As mentioned in Section~\ref{sub:scheme:inlining_apply_nodes},
Waterman~\cite{AdaptvCompilAndInlingWaterman} implemented a hillclimber
algorithm to adaptively re-compile a program for finding the most decent
inlining heuristic for each compilation.

It is our belief that the strengths of Jive's RVSDG approach to
compilation/inlining can be further exploited if the choosing of inlining
heuristic is done adaptively like Waterman proposes.

The work of Cooper, et al.~\cite{AdaptvStratInlSubst} shows how the
parametrization of Waterman's~\cite{AdaptvCompilAndInlingWaterman} inliner can
easily be expanded upon and added to, making it a more powerful tool. This is
also something we believe that Jive could benefit from wrt. to inlining, but was
neither part of this project's scope, nor did we have the resources to implement
it.

\subsection{Choosing loop breakers more carefully}
\label{sub:fw:optimal_loop_breakers}

When deciding upon recursive functions to be marked as \textit{loop breakers},
discussed in Section~\ref{sub:scheme:inlining_recur_apply_nodes}, our approach
does not choose these carefully. Ideally, an approach such as the one
den~Heijer~\cite{BasMscThesis} proposes, should be utilized because it makes an
effort in choosing the minimum amount of loop breakers needed. This enables a
larger amount of the functions residing in recursive environments to be inlined.

Again, we believe the addition of using den~Heijer's approach based on the
Directed Blackout Feedback Vertex Set problem would have been beneficial to
implement in Jive's inliner.

\subsection{The order of inlining}
\label{sub:fw:call_site_visit_order}

To our knowledge, no other literature exist on the subject of the consequences
of the ordering of inlined functions. As illustrated with the example from
Section~\ref{sub:scheme:ordering_apply_nodes}, the order in which inlines are
performed affects not only the total count of inlines, but also \textit{which}
call sites are inlined.

Since the RVSDG gives us a directed call graph for each program compiled, other
approaches can be tested. One such approach could be to see whether there are
inherent properties discovered when ordering the call sites in a top-down
ordering as opposed to a bottom-up ordering.

Another approach could be to find the aggregate cost of inlining nested function
calls: profiling them first, before inlining them. This could perhaps become
useful, if not even more effective, when the inliner has a long chain of nested
function calls. If the sum of inlining all the function calls in the chain
exceeds the limits set by the inliner heuristic, could the profiling help find
out which parts of the chain are most advantageous to inline first?

While interesting food for thought, this is outside the scope of our project,
and hence we have not done more implement a feature permitting our inliner to
visit the nodes in a top-down or bottom-up order. Section~\ref{sub:res:inlining}
reports the difference when we changed the order of inlining.
