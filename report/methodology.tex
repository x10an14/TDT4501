% !TEX root = ./report.tex

\clearpage
\section{Methodology}
\label{sec:methodology}

Using the \textit{Inliner Conditions} (ICs) described in
Section~\ref{sub:meth:inlining_conditions}, the inliner of this project has been
tested using C-language programming files from the SPEC2006 Benchmark Suite.
This section first details the ICs and the rationale behind CNF clauses used for
testing, before explaining how the files tested from the SPEC2006 Benchmark
Suite were chosen.

\subsection{The Inlining Conditions}
\label{sub:meth:inlining_conditions}

The ICs utilized in this project are as follow:

\begin{itemize}

	\item \nopagebreak{\textbf{Node Count }(NC)}

This function property represents the number of operations (nodes) contained
within a function. A function's node count is an IC we want to utilize because
it gives us an indication for what the upper limit of potential code duplication
might become, if we inline the function.

	\item \nopagebreak{\textbf{Loop Nesting Depth }(LND)}

This call site property tells us how potentially useful it is to inline this
specific call site. The assumption is that most of a program's execution time is
spent within loops, so there is potentially more to gain if optimizations are
unveiled by inlining call sites inside nested loops.

	\item \nopagebreak{\textbf{Static Call Count }(SCC)}

This property tells us how many call sites invoke this function in the program.
If this count is low, it may be worth inlining all the call sites and
eliminating the original function. For example, if the count is $1$, and the
function is not exported, then the call site can always be inlined since there
is no risk of code duplication.

	\item \nopagebreak{\textbf{Parameter Count }(PC)}

The greater the amount of parameters a function has, the greater the invocation
cost of said function. This can especially be the case when type conversion is
required~\cite{AdaptvCompilAndInlingWaterman}. In some cases, the computational
cost of an inlined function with low node count may be smaller than the cost of
invoking it if it has many parameters~\cite{AdaptvCompilAndInlingWaterman}.

	\item \nopagebreak{\textbf{Constant Parameter Count }(CPC)}

This property tells us how many of the call site's parameters are constant at
the call site. Function invocations with constant parameters can often benefit
more from unveiled optimizations after inlining.

	\item \nopagebreak{\textbf{Calls In Node }(CIN)}

This function property tells us how many call sites are located inside the
function the call site being evaluated invokes. Hence, it enables the finding of
leaf functions. Waterman~\cite{AdaptvCompilAndInlingWaterman} introduced this
parameter for two distinct reasons: leaf functions are often small and easily
inlined, and a high percentage of total execution time is spent in leaf
functions.

	\item \nopagebreak{\textbf{Exported}(EXP)}

This inlining condition tells us whether or not the function called is exported
or not. If it is not, it is of greater interest to inline, because if all call
sites are inlined, the declaration of the function can be removed through
\textit{Dead Code Elimination}.

\end{itemize}

\subsection{The CNF}
\label{sub:meth:cnf}

Section~\ref{sub:scheme:inlining_apply_nodes} explains the generic form a CNF
may have, and how our inlining conditions introduced in
Section~\ref{sub:meth:inlining_conditions} can help build such.

While a certain CNF may prove quite helpful with one program, there is no
guarantee that other programs would benefit equally, if at all, from the same
CNF. In fact, a CNF which proves to be the best for the execution time of one
program, may increase the execution time of another.

Hence, care must be taken when selecting a CNF. Since today's compilers are
expected to run in seconds or less, it is a difficult challenge finding
compromises between a CNF that gives decent results for many different
applications, yet is efficient enough not to slow down the compilation time too
much. The potential code size increase of the compiled program cannot be
forgotten either.

With these limitations in mind, and through some inspiration from GCC
Waterman~\cite{AdaptvCompilAndInlingWaterman}, we decided on a CNF built from
the following clauses:

\begin{enumerate}

	\item \lstinline!EXP == false && SCC == 1!

What this clause asks for is whether or not the function being called is not
exported, and that there are no other call sites invoking this function. If so,
then there is nothing to lose by inlining it. Not only is the function
invocation cost removed, but there is no risk of negatively affecting the size
of the compiled end result. This most often occurs after enough call sites have
been inlined, so that \lstinline|SCC| becomes \lstinline|1|.

	\item \lstinline!NC < X!
	\label{sub:meth:cnf_clause2}

This clause only asks whether or not the amount of operations inside the
function being invoked are less than the placeholder value \lstinline!X!. This
is to remove all small nodes which perform simple tasks, such as getter's,
setter's, increments, and so forth. The code size does not run a big risk of
being negatively affected by inlining these. In fact, inlining such function can
conceivably increase potential optimizations such as \textit{Common
Subexpression Elimination}.

	\item \lstinline!CIN < Y!
	\label{sub:meth:cnf_clause3}

With the intention of having the placeholder \lstinline!Y! at lower values, this
clause attempts to seek out the leaf nodes in nested calls.
Research~\cite{AdaptvCompilAndInlingWaterman} shows that a lot of execution time
are spent in these, and hence they are of interest to inline, in the hope that
other optimization techniques may be applied on their operations.

	\item \lstinline!NC < Z && CPC > V!
	\label{sub:meth:cnf_clause4}

This clause wants to inline call sites whose constant parameters exceed the
placeholder value \lstinline!V!. The term \lstinline|NC < Z| is there to prevent
code size explosion. limiting the inlining to functions whose node count is less
than \lstinline|Z|. If the input parameters of the function are constant, then
there's conceivably a bigger potential for other optimization techniques to be
applied, if inlined.

	\item \lstinline!NC < W && LND > T!
	\label{sub:meth:cnf_clause5}

The final clause is very similar to the previous one, but instead of looking for
call sites with constant parameters, it looks for call sites residing inside of
loops. It is well known that most of the execution time of a program is spent
inside of loops. This makes the unveiling of potential optimizations of function
calls inside of loops tempting. Again, to avoid code size explosion, and to help
minimize the placeholder value \lstinline|T|, no functions with internal node
count larger than \lstinline|W| are inlined by this clause.

\end{enumerate}

Thus, the final CNF has the form:

\begin{centering}
\lstinline!(EXP == true && SCC = 1)|| NC < Z || CIN < Y ||! \\
\lstinline!(NC < Z && CPC > V)|| (NC < W && LND > T)! \\
\end{centering}

See Section~\ref{sub:res:final_cnf} for the final form with the placeholder
values chosen.

\subsection{The SPEC2006 Benchmarks}
\label{sub:meth:SPEC2006_files}

The SPEC2006 Benchmark Suite files were chosen with the following criteria:

\begin{enumerate}
	\item \nopagebreak{The Benchmark Suite program code files were written in C.}

	\item \nopagebreak{Clang}-3.3 (on Ubuntu 14.04) was able to convert the inputted C files
to LLVM IR assembly code with the \lstinline!-S! and \lstinline!-emit-llvm!
flags.

	\item \nopagebreak{Jive was able to interpret all of the assembly instructions in the}
.ll files generated by Clang and construct RVSDGs from them.

	\item \nopagebreak{The files could to be tested within a time limit of 120}
seconds\footnote{This due to a bug in the library needed for compiling an
executable using Jive.}, executed single-process, on a Intel(R) Core(TM)
i5-4200M CPU @ 2.50GHz, with 3072 KB cache size, Ubuntu 14.04 64-bit based Linux
distribution.
\end{enumerate}

With these requirements, files from the following benchmarks were used for
testing:

\todo[inline]{Fill table with final numbers.}

\begin{table}[h]
\centering
\begin{tabular}{|
>{\columncolor[HTML]{EFEFEF}}l |l|l|l|l|l|l|l|l|l|l|l|l|l|}
\hline
\rotatebox{60}{Benchmarks} & \rotatebox{90}{perlbench} & \rotatebox{90}{bzip2} & \rotatebox{90}{gcc} & \rotatebox{90}{mcf} & \rotatebox{90}{milc} & \rotatebox{90}{gromacs} & \rotatebox{90}{gobmk} & \rotatebox{90}{hmmer} & \rotatebox{90}{sjeng} & \rotatebox{90}{libquantum} & \rotatebox{90}{h264ref} & \rotatebox{90}{lbm} & \rotatebox{90}{sphinx3} \\ \hline
\# files  &           &       &     &     &      &         &       &       &       &            &         &     &         \\ \hline
\end{tabular}
\end{table}

See Appendix~\ref{app:SPEC2006_files_used} for the complete list of .c files
used for testing from each of the above benchmarks from SPEC2006.
