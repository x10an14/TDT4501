% !TEX root = ./report.tex

\clearpage
\section{Methodology}
\label{sec:methodology}

\todo[inline]{Need an introduction here, no?}

\subsection{Inlining conditions used when inlining}
\label{sub:meth:inlining_conditions}

To effectively test for an apt heuristic when deciding whether or not to inline
a call site\footnote{As discussed in
Section~\ref{sub:scheme:inlining_apply_nodes}.}, our approach is based on
previous work\cite{deshpande2012statically}\cite{AdaptvCompilAndInlingWaterman}.
This approach utilizes something we call  \textit{Inliner Conditions} (ICs)
which evaluate the function invoked by the call site.

Using ICs in this way allows us to write and re-write inlining heuristics
effectively, since we can write them using CNF in the following fashion:
\lstinline"SC < X || SCC < Y || (SCC < Z && LND > W)"

The ICs utilized in this project are the following:

\begin{itemize}

	\item Statement count (SC): This function property is equates to the number of
C/C++ statements contained within a function.A function's statement count is an
inliner condition we want to utilize because it is gives us an idea of the size
of the code- duplication if we inline the function.

	\item Loop nesting depth (LND): This property tells us how potentially useful it
is to inline this specific call site. The assumption is that most of a program's
execution time is spent within loops, so there is potentially more to gain if
optimizations are unveiled by inlining call sites inside nested loops.

	\item Static call count (SCC): This property tells us how many call sites there
are for this function in the program. If this count is low, it may be worth
inlining all the call sites and eliminating the original function. If the count
is $1$, then the call site can always be inlined, seeing as there is no risk of
code-duplication.

	\item Parameter count (PC): The greater the amount of parameters a function has,
the greater the invocation cost of said function. This is especially true when
type conversion is required. In some cases, the computational cost of an inlined
with low statement count may be smaller than the cost of invoking it if it has
many parameters\cite{AdaptvCompilAndInlingWaterman}.

	\item Constant parameter count (CPC): This property tells us how many of the call
site's parameters are constant at the call site. Function invocations with
constant parameters can often benefit more from unveiled optimizations after
inlining.

	\item Calls in procedure (CP): This function property tells us how many call
sites are located inside the function the call site invocates. Hence, it enables
finding leaf functions. Waterman\cite{AdaptvCompilAndInlingWaterman} introduced
this parameter for two distinct reasons: leaf functions are often small and
easily inlined, and a high percentage of total execution time is spent in leaf
functions.

\end{itemize}
