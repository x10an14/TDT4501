% !TEX root = ./report.tex

\section{Related Work}

In this section we first explain why inlining is a practice found in almost
every compiler to date, before we list (in order of relevance)\todo{To do?} the
papers contributing to the background of this report, and finally summarize what
each paper brings to this report.

\subsection{Unread (but found interesting) papers}

\begin{itemize}
	\item ``\textbf{On building a Supercompiler for GHC}''', Peter A. Jonsson
and Johan Nordlander, at Luleå University of Technology.
\url{http://pure.ltu.se/portal/files/2231262/nwpt08-scp.pdf}
	\item ``\textbf{Automatic Tuning of Inlining Heuristics}'', written by John
Cavazos and Michael F.P. O'Boyle, at School of Informatics University of
Edinburg.
\url{http://ieeexplore.ieee.org/xpls/abs_all.jsp?arnumber=1559966}
	\item ``\textbf{Optimizing Generics Is Easy!}'', written by José Pedro
Magalhães and Stefan Holdermans and Johan Jeuring and Andres Löh, at Utrecht
University, The Netherlands.
\url{http://www.andres-loeh.de/OptimizingGenerics/}
\textit{Looks more like a description of how GHC inlines than anything else to
do with inlining.}
	\item ``\textbf{Inline expansion: when and how?}'', Manuel Serrano, at
University of Geneva, Switzerland.
\url{http://dl.acm.org/citation.cfm?id=692960}
\textit{Most interesting of this list at the moment.}
	\item ``\textbf{Region-based compilation: an introduction and motivation}'',
Richard E. Hank and Wen-Mei W. Hwu, at University of Illinois, and B.
Ramakrishna Rau, at Hewlett Packard Laboratories, Plo Alto California.
\url{http://dl.acm.org/citation.cfm?id=225160.225189&coll=DL&dl=ACM&CFID=470941312&CFTOKEN=25125832}
\textit{Second most interesting on this list at the moment.}
	\item ``\textbf{Bigloo: a portable and optimizing compiler for strict
functional languages}'', by Manuel Serrano and Pierre Weis.
\url{http://citeseerx.ist.psu.edu/viewdoc/summary?doi=10.1.1.50.8424}
	\item ``\textbf{Losing functions without gaining data: another look at
defunctionalisation}'', by Neil Mitchell and Colin Runciman, at University of
York, UK.
\url{http://dl.acm.org/citation.cfm?id=1596641}
	\item ``\textbf{Statically Unrolling Recursion to Improve Opportunities for
Parallelism}'', by Neil Deshpande and Stephen A. Edwards, at Columbia University
NY.
\url{http://www.cs.columbia.edu/~sedwards/papers/deshpande2012statically.pdf}
\end{itemize}

\subsection{Read papers}

\begin{enumerate}[1)]
	\item ``\textbf{Secrets of the Glasgow Haskell Compiler inliner}'', written
by Simon Peyton Jones and Simon Marlow, at Microsoft. \\
This paper explains how inlining is dealt with by the GHC compiler. It spreads
this into the following sections:
\begin{enumerate}[a)]
	\item Short section describing what inlining is, and its most obvious and
generic properties and pitfalls (Section 2).
	\item An explanation of a strategy of how (and when) to inline recursive
functions (Section 3).
	\item How they so deal with name capture (Section 4).
	\item As well as discussing certain moments/situations when inlining is
carefully considered by the GHC. They also have a section describe how they
exploit their name-capture solution to support accurate tracking of both lexical
and evalutation-state environments (Section 5+6).
	\item Finally they sketch their implementation (Section 7).
\end{enumerate}


\end{enumerate}
