% !TEX root = ./report.tex

\section{Related Work}

In this section (...)\todo[inline]{To do...}

\subsection{Unread (but found interesting) papers}

\begin{itemize}
	\item ``\textbf{On building a Supercompiler for GHC}''', \\ Peter A. Jonsson
and Johan Nordlander, at Luleå University of Technology. \\
\url{http://pure.ltu.se/portal/files/2231262/nwpt08-scp.pdf}
	\item ``\textbf{Optimizing Generics Is Easy!}'', \\ written by José Pedro
Magalhães and Stefan Holdermans and Johan Jeuring and Andres Löh, at Utrecht
University, The Netherlands.
\textit{Looks more like a description of how GHC inlining than anything else on
the subject of inlining.}\\
\url{http://www.andres-loeh.de/OptimizingGenerics/}
	\item ``\textbf{Should potential loop optimizations influence inlining
decisions?}'', by Christopher Barton and José Nelson Amaral and Bob Blainey, at
University of Alberta and IBM Toronto Software Laboratory, Canada.
\textit{Perhaps most 2n interesting at this moment?}\\
\url{http://dl.acm.org/citation.cfm?id=961329}
	\item ``\textbf{Region-based compilation: an introduction and motivation}'',
\\ byRichard E. Hank and Wen-Mei W. Hwu, at University of Illinois, and B.
Ramakrishna Rau, at Hewlett Packard Laboratories, Plo Alto California.
\textit{3rd most interesting on this list at the moment.} \\
\url{http://dl.acm.org/citation.cfm?id=225160.225189&coll=DL&dl=ACM&CFID=470941312&CFTOKEN=25125832}
	\item ``\textbf{Bigloo: a portable and optimizing compiler for strict
functional languages}'', \\ by Manuel Serrano and Pierre Weis.
\textit{Does not really seem interesting/relevant, but Manuel Serrano has
written several relevant papers, and strict functional languages are some of
the more viable languages for this type of optimization...} \\
\url{http://citeseerx.ist.psu.edu/viewdoc/summary?doi=10.1.1.50.8424}
	\item ``\textbf{Losing functions without gaining data: another look at
defunctionalisation}'', \\ by Neil Mitchell and Colin Runciman, at University of
York, UK.
\textit{Not really sure what to make of this one.} \\
\url{http://dl.acm.org/citation.cfm?id=1596641}
	\item ``\textbf{Statically Unrolling Recursion to Improve Opportunities for
Parallelism}'', \\by Neil Deshpande and Stephen A. Edwards, at Columbia
University NY.
\textit{This one sounds cool, and being static is a plus. But that's all its got going for it without having read it more closely. It does discuss
inlining.}\\
\url{http://www.cs.columbia.edu/~sedwards/papers/deshpande2012statically.pdf}
	\item ``\textbf{An adaptive strategy for inline substitution}'', \\ by Keith
D. Cooper and Timothy J. Harvey and Todd Waterman, at Rice University Houston
and Texas Instruments Inc. Stafford, Texas.
\textit{Perhaps most interesting at this time due to discussing global/static
optimizations as well as being relatively new compared to other papers in this
list?}\\
\url{http://dl.acm.org/citation.cfm?id=1788381}
\end{itemize}

\subsection{Read papers}

\begin{enumerate}
	\item ``\textbf{Secrets of the Glasgow Haskell Compiler inliner}'',
\cite{GHC-paper}

This paper explains how inlining is dealt with by the GHC compiler. It spreads
this into the following sections:
\begin{enumerate}
	\item Short section describing what inlining is, and its most obvious and
generic properties and pitfalls (Section 2).
	\item An explanation of a strategy of how (and when) to inline recursive
functions (Section 3).
	\item How they so deal with name capture (Section 4).
	\item As well as discussing certain moments/situations when inlining is
carefully considered by the GHC. They also have a section describe how they
exploit their name-capture solution to support accurate tracking of both lexical
and evalutation-state environments (Section 5+6).
	\item Finally they sketch their implementation (Section 7).
\end{enumerate}

	\item ``\textbf{Inline expansion: when and how?}'',
\cite{InlineWhenHowSerrano}

This paper focuses on \textit{when} and \textit{how} to inline functions,
classified as either \textit{recursive} or \textit{non-recursive} functions.
This is done in the programming language Scheme, with \todo{confirm this?}the
Bigloo compiler M. Serrano has created.

The paper gives an algorithm for when to inline (which \cite{GHC-paper} remarks
upon), as well as comments and results on their inlining wrt. the two classes of
functions the paper defines.

	\item ``\textbf{Automatic Tuning of Inlining Heuristics}'',
\cite{AutoTuningJavaHeuristics}

This paper explores and explains the advantages of genetic algorithm  heuristics
when dynamically compiling Java. Since it is dynamically compiled, it's harder
for the compilation unit to get a proper global view of the compilation job,
required resources, and beneficial trade-offs. All of which play important roles
when deciding whether to incur the cost of increased code size when inlining.


\end{enumerate}
