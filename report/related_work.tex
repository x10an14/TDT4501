% !TEX root = ./report.tex

\section{Related Work}

In this section (...)\todo[inline]{To do...}

\subsection{Read papers}

Below papers are listed according to perceived relevance/usefulness for this
project.\\

\begin{enumerate}
	\item ``\textbf{Secrets of the Glasgow Haskell Compiler inliner}'',
\cite{GHC-paper}

This paper explains how inlining is dealt with by the GHC compiler. It spreads
this into the following sections:
\begin{enumerate}
	\item Short section describing what inlining is, and its most obvious and
generic properties and pitfalls (Section 2).
	\item An explanation of a strategy of how (and when) to inline recursive
functions (Section 3).
	\item How they so deal with name capture (Section 4).
	\item As well as discussing certain moments/situations when inlining is
carefully considered by the GHC. They also have a section describe how they
exploit their name-capture solution to support accurate tracking of both lexical
and evalutation-state environments (Section 5+6).
	\item Finally they sketch their implementation (Section 7).
\end{enumerate}

	\item ``\textbf{Inline expansion: when and how?}'',
\cite{InlineWhenHowSerrano}

This paper focuses on \textit{when} and \textit{how} to inline functions,
classified as either \textit{recursive} or \textit{non-recursive} functions.
This is done in the programming language Scheme, with \todo{confirm this?}the
Bigloo compiler M. Serrano has created.

The paper gives an algorithm for when to inline (which \cite{GHC-paper} remarks
upon), as well as comments and results on their inlining wrt. the two classes of
functions the paper defines.

	\item ``\textbf{Automatic Tuning of Inlining Heuristics}'',
\cite{AutoTuningJavaHeuristics}

This paper explores and explains the advantages of genetic algorithm  heuristics
when dynamically compiling Java. Since it is dynamically compiled, it's harder
for the compilation unit to get a proper global view of the compilation job,
required resources, and beneficial trade-offs. All of which play important roles
when deciding whether to incur the cost of increased code size when inlining.


\end{enumerate}
