% !TEX root = ./report.tex

\clearpage
\section{Related Work}

As mentioned in Section \ref{introduction}, compilers have existed, and
optimized code, since the last half of the 20th century. Inlining has long been
an important optimization for most compilers. W. Davidson and M. Holler
\cite{SubprogInlining} examine the hypothesis that the increased code size of
inlined code affects execution time on demand-paged virtual memory machines.
Using equations developed to describe the execution time of an inlined program,
they test this hypothesis through the use of a source-to-source subprogram
inliner.

Cavazos and F.P. O'Boyle \cite{AutoTuningJavaHeuristics} use a genetic algorithm
in their auto-tuning heuristics to show how conjunctive normalform (CNF) can
easily be used to decide if and when to inline a specific call site. They report
between 17\% and 37\% execution time improvements without code size explosion.

Serrano \cite{InlineWhenHowSerrano} implements an inliner in the Scheme
programming language. The paper details an heuristic for which functions to
inline, as well as an algorithm for how to inline recursive functions. The paper
reports an average run time decrease of 15\%.

\todo[inline]{Insert HiPEAC paper between these two?}

Waterman's Ph.D. thesis \cite{AdaptvCompilAndInlingWaterman} examines the use of
adaptive compilation techniques in combination with an inlining heuristic. His
thesis shows how CNF can be used for deciding which functions to inline. It also
details how there can be no single given correct set of parameters for all
programs, given the search space the heuristics hillclimbing algorithm
uses\footnote{To decide which functions to inline.}. The thesis reports
consistently better or equal run time compared to the GCC inliner and ATLAS.

D. Cooper et. al \cite{AdaptvStratInlSubst} expand on Waterman's PhD Thesis
\cite{AdaptvCompilAndInlingWaterman}. Their paper details how the proper use of
the parameterization search space using a hillclimber algorithm, in an adaptive
inlining scheme can achieve improved results compared to GCCs inliner. Their
results range from 4\% to 31\% run time decrease compared to GCCs inliner.

E. Hank et. al \cite{RegionBasedCompilationIntroduction} introduce a new
technique called \textit{Region-Based Compilation}. They examine the benefits an
aggressive compiler gains from inlining on Very Long Instruction Word (VLIW)
architectures. The paper reports that aggresive inlining can become costly, with
an average code size expansion of 400\%. However, their results also show that
inlining is sufficiently able to unveil further compiler optimizations, leading
to an average of 50\% of program execution time spent in functions with more
than 1000 operations. This is an improvement, compared to their test result of
more than 80\%, of execution time spent inside functions with less than 250
operations when run without inlining.

P. Jones and Marlow \cite{GHCPaper} describe the inliner for the Glasgow Haskell
Compiler (GHC). Their paper introduces a novel approach for deciding which
mutually recursive functions can safely be inlined without code size explosion
or the risk of non-termination. Jones and Marlow report on average of 30\% run
time decrease.

The report of Barton et. al \cite{ShouldLoopOptsInfluenceInlining} tests whether
the potential for loop fusion should be taken into consideration in the inliner.
They disprove this using the IBM\textregistered XL Compile Suite, measuring how
many additional loops they were able to fuse in the SPECint2000 and SPECfp2000
benchmark suites. The results reported indicate that the compiler already
catches most of the potential loop fusion optimizations, and the results cannot
justify an inter-procedural loop fusion implementation.

Deshpande and A. Edwards \cite{deshpande2012statically} detail an inlining
algorithm meant to improve inlining in the GHC. The algorithm improved the
parallelism of recursive functions by ``widening'' them, into the equivalent of
multiple recursive calls through unrolling recursion. No results were reported.

W. Hwu and P. Chang \cite{InlineFuncExpCProgs} explore how program profile
information could be used to decide whether or not to statically inline C
functions. Their motivation was to remove costly function calls in a C program,
in addition to unveil potential optimizations. Through the use of the IMPACT-I C
compiler, they profile dynamic program information, resulting in a call graph
with weighted edges. They report 0\% to 99\% reduction of dynamic function calls
in their test benchmarks.
