% !TEX root = ./report.tex

\section{Related Work}

In this section we first explain why inlining is a practice found in almost
every compiler to date, before we list (in order of relevance)\todo{To do?} the
papers contributing to the background of this report, and finally summarize what
each paper brings to this report.

\subsection{Inlining}
\todo[inline]{Decide whether to keep or throw out.}

\subsection{Read papers}

\begin{enumerate}[1)]
	\item ``\textbf{Secrets of the Glasgow Haskell Compiler inliner}'', written
by Simon Peyton Jones and Simon Marlow, at Microsoft. \\

This paper explains how inlining is dealt with by the GHC compiler. It spreads
this into the following sections:
\begin{enumerate}[a)]
	\item Short section describing what inlining is, and its most obvious and
generic properties and pitfalls (Section 2).
	\item An explanation of a strategy of how (and when) to inline recursive
functions (Section 3).
	\item How they so deal with name capture (Section 4).
	\item As well as discussing certain moments/situations when inlining is
carefully considered by the GHC. They also have a section describe how they
exploit their name-capture solution to support accurate tracking of both lexical
and evalutation-state environments (Section 5+6).
	\item Finally they sketch their implementation (Section 7).
\end{enumerate}

	\item ``\textbf{Automatic Autotuning of Inlining Heuristics}'', written by
John Cavazos and Michael F.P. O'Boyle, at School of Informatics University of
Edinburg. \\
	\todo[inline]{Summary of paper.}

\end{enumerate}
