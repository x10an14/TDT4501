% !TEX root = ./report.tex

\section{Related Work}

As mentioned in Section \ref{introduction}, inlining has been done since the
last half of the 20th century. Following comes the related work we came across
researching for the Jive's inliner.

\paragraph{}

W. Davidson and M. Holler \cite{SubprogInlining} examine the hypothesis that
the increased codesize of inlined code affects the execution time performance on
demand-paged virtual memory machines. Using equations developed to the describe
an inlined programs' execution time, they test this hypothesis through the use
of a source-to-source subprogram inliner.

Cavazos and F.P. O'Boyle \cite{AutoTuningJavaHeuristics} use a genetic algorithm
in their auto-tuning heuristics to show how conjunctive normalform (CNF) can
easily be used to decide if and when to inline a specific call site. They report
between 17\% and 37\% execution time improvements without codesize explosion.

Serrano \cite{InlineWhenHowSerrano} implements an inliner in the Scheme
programming language. He details an algorithm for which functions to inline, as
well as an algorithm for how to inline recursive functions and non-recursive
functions.

Waterman's Ph.D. thesis \cite{AdaptvCompilAndInlingWaterman} examines the use of
techniques to adaptively decide which functions to inline. The thesis shows the
use of CNF for deciding which functions to inline.

E. Hank, W. Hwu, and R. Rau \cite{RegionBasedCompilationIntroduction} introduces
a new technique called \textit{Region-Based Compilation}. And examines the
benefits an aggressive compiler can gain from inlining.

P. Jones and Marlow \cite{GHCPaper} explore an inlining approach for the Glasgow
Haskell Compiler (GHC). The paper introduces a novel approach for deciding what
mutually recursive functions can be safely inlined without code explosion or
non-terminating programs.

Barton, N. Amaral, and Blainey \cite{ShouldLoopOptsInfluenceInlining} tests
whether there should be put more effort into making better inlining decisions,
with the intent on helping the compilers improve on loop optimizations.

Deshpande and A. Edwards \cite{deshpande2012statically} detail how inlining
should be done in the GHC, how it is useful, and the correctness of their
presented algorithm.

W. Hwu and P. Chang \cite{InlineFuncExpCProgs} explored in 1989 how \todo{Can I
say this? Should it be ``function profile (...)'' instead?}program profile
information could be used to decide whether or not to inline C functions
statically. Their motivation was to remove costly function calls in  a C
program, in addition to statically unveil potential optimizations.

D. Cooper, J. Harvey, and Waterman \cite{AdaptvStratInlSubst} build work much
similar to Waterman's PhD Thesis \cite{AdaptvCompilAndInlingWaterman}. Their
paper looks into how parameterization of an adaptive inlining scheme can achieve
great results compared to todays typical inlining schemes\footnote{Comparison
made mainly with GCC.}

\subsection{Regionalized Value-State Dependency Graph}
\todo[inline]{Insert reference/summary of HiPEAC paper when published}
