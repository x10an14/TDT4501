% !TEX root = ./report.tex

\section{Related Work}

In this section (...)\todo[inline]{To do...}

\subsection{Inlining}

W. Davidson and M. Holler \cite{SubprogInlining} examine the proposition that
the increased codesize of inlined code affects the execution time performance on
demand-paged virtual memory machines. Using equations developed to the describe
an inlined programs' execution time, they test this propositing through the use
of a source-to-source subprogram inliner.

Cavazos and F.P. O'Boyle \cite{AutoTuningJavaHeuristics} use a genetic algorithm
in their auto-tuning heuristics to show how conjunctive normalform (CNF) can
easily be used to decide if and when to inline a specific call site. They report
between 17\% and 37\% execution time improvements without an explosion of the resulting code size.

Serrano \cite{InlineWhenHowSerrano} implements an inliner in the Scheme
programming language. He details an algorithm for which functions to inline, as
well as an algorithm for how to inline reccursive functions and non-recursive
functions.

Waterman's Ph.D. thesis \cite{AdaptvCompilAndInlingWaterman} examines the use of
techniques to adaptively decide which functions to inline. The thesis shows the
use of CNF for deciding which functions to inline.

E. Hank, W. Hwu, and R. Rau \cite{RegionBasedCompilationIntroduction} introduces
a new technique called \textit{Region-Based Compilation}. And examines the
benefits an aggressive compiler can gain from inlining.

P. Jones and Marlow \cite{GHCPaper} explore a inlining approach for the Glasgow
Haskell Compiler (GHC). The paper introduces a novel approach for deciding which
mutually recursive functions can be (if any) safely inlined for optimization
purposes.

\subsection{Regionalized Value-State Dependency Graph}
\todo[inline]{Insert reference/summary of HiPEAC paper when published}
