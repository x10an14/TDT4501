% !TEX root = ./report.tex

\clearpage
\section{Results}
\label{sec:res}

In this section, we first profile the Benchmarks introduced in
Section~\ref{sec:methodology}, and then more specifically the \applyNode s in
the Benchmark Suite, before we explain the reasoning behind the placeholder
values we decided upon for the CNF we used in our testing. Thereafter we show
and discuss the results of our testing. Some focus is put on the status of Jive,
and how its status may have affected our test results.

\subsection{Profiling SPEC2006 and its functions}
\label{sub:res:profiling}

To decide upon the placeholder values for the the CNF introduced in
Section~\ref{sub:meth:cnf}, we profiled the SPEC2006 Benchmarks for the
following properties among others:

First and foremost, while a higher node count may still result in faster
execution in some cases, minimizing the amount of operations in a programs'
RVSDG was one of our main motivations behind choosing the placeholder values of
the CNF.

\begin{centering}
	\noindent\begin{minipage}{\textwidth}
		\captionsetup{type=figure}
		\hspace{-1em}
		\includegraphics[width=\textwidth]{figures/gnuplot/avg_node_count_pre_inlining}
	\end{minipage}
	\captionof{figure}{Histogram showing the average amount nodes present in
each SPEC2006 Benchmark. The y-axis is broken due the two benchmark's \textit{
h264ref} and \textit{lbm}'s big difference in average node count.}
	\label{fig:benchmarks_avg_nc_pre_inlining}
\end{centering}

Figure~\ref{fig:benchmarks_avg_nc_pre_inlining} shows the (linearly) averaged
node count for each of the SPEC2006 Benchmarks used. We use
Figure~\ref{fig:benchmarks_avg_nc_pre_inlining} as our baseline, to which we
compare the results of our testing in Section~\ref{sub:res:inlining}.

However, we want to explain the reason behind \textit{lbm}'s high node count
average: it is caused by one of \textit{lbm}'s files which contains functions
that fill/modify large \lstinline!IO_FILE!s with content. \textit{h264ref} also
stands out, though not as much as \textit{lbm}. It stands out due to its low
average of nodes per file. It does so because it only has one file which made it
past the requirements listed in Section~\ref{sub:meth:SPEC2006_files},
specifically requirement 3, and the majority of this file consists of two leaf
functions.

This observation raises the obvious point that by profiling the functions in the
benchmarks, we can observe how many of the functions are exported, seeing as
Jive is unable to link the program files at compile time. This would give us an
upper bound for how many functions we can remove through \textit{Dead Code
Elimination} (DCE). Figure~\ref{fig:avg_lambda_count_pre} shows us how many functions
there are on average in each Benchmark, and how big of an average of these are
exported.

\begin{centering}
	\noindent\begin{minipage}{\textwidth}
		\captionsetup{type=figure}
		\hspace{-1em}
		\includegraphics[width=\textwidth]{figures/gnuplot/avg_lambda_count_pre}
	\end{minipage}
	\captionof{figure}{Histogram showing that there is a small number of
functions we can hope to eliminate through DCE. The left bar shows the average
count of $\lambda$-nodes per benchmark, while the right bar shows the average
count of exported $\lambda$-nodes in the same benchmark.}
	\label{fig:avg_lambda_count_pre}
\end{centering}

One of the IC most used in our final CNF form is \textit{Node Count} (NC). As
such, it is of use to us to know what NC functions have on average in our
SPEC2006 test files. Figure~\ref{fig:benchmarks_avg_nc_in_lambda_per_file}
depicts this in a scatter plot.

\begin{centering}
	\noindent\begin{minipage}{\textwidth}
		\captionsetup{type=figure}
		\hspace{-1em}
		\includegraphics[width=\textwidth]{figures/gnuplot/avg_nc_in_lambdas_pre}
	\end{minipage}
	\captionof{figure}{Scatter plot showing the spread of average amount of nodes
contained within each function per file in the Benchmarks Suite. The files are
grouped by Benchmark along the horizontal axis.}
	\label{fig:benchmarks_avg_nc_in_lambda_per_file}
\end{centering}

The spread of data points in
Figure~\ref{fig:benchmarks_avg_nc_in_lambda_per_file} tells us that while there
are some functions like the aforementioned of \textit{gcc}, the majority of them
contain less than 100 nodes. Which conforms well with the stipulation that most
applications majority of nodes are leaf nodes, which generally contain few
operations. This makes it somewhat easier for us to decide what limits NC should
present with regards to inlining, since we can see that a majority would be
inlined if we inlined all functions satisfying \lstinline|NC < 200|.

The histogram in Figure~\ref{fig:phis_and_lambdas_in_phis} tells us that there
are no recursive environments containing more than one function in the
Benchmarks tested. In other words, that all the recursive functions are
\nolinebreak{self-recursive}. As such, we did not get to test our algorithm
detailed in Section~\ref{sub:scheme:inlining_recur_apply_nodes} with the
SPEC2006 Benchmark Suite, but in our own testing we did confirm that it
permitted us to safely inline \textit{some} functions inside a recursive
environment containing multiple functions.

\begin{centering}
	\noindent\begin{minipage}{\textwidth}
		\captionsetup{type=figure}
		\hspace{-1em}
		\includegraphics[width=\textwidth]{figures/gnuplot/avg_phis_and_lambdas_in_phis}
	\end{minipage}
	\captionof{figure}{Histogram showing the average amount of $\phi$-regions
per benchmark with the left bar. The left bar shows the average amount of
$\lambda$-nodes inside each $\phi$-region present in a benchmark.}
	\label{fig:phis_and_lambdas_in_phis}
\end{centering}

Another property of functions of interest to us is the average amount of call
sites contained within functions. This property gives us an idea as to the
proportion of how many leaf nodes there are with no function calls in a
benchmark, as well as the proportion of functions with $X$ amount of call sites
on average.

Figure~\ref{fig:benchmarks_avg_cin_lambda_per_file} shows us that on average the
balance between functions with no call sites, and functions with, is not sharply
skewed one way or the other. We do have some outliers, like the aforementioned
files from \textit{gcc}, but one can safely say from the plot that a majority of
the functions have less than two hundred call sites contained within.

The histogram in Figure~\ref{fig:benchmarks_avg_cin_lambda_per_file} shows that
beneath the average of two hundred call sites in function mark on the y-axis,
there is a slight predominance for functions to have about one hundred or less.
These observations are helpful in giving us upper bounds for how many functions
could potentially be inlined by this IC.

\begin{centering}
	\noindent\begin{minipage}{\textwidth}
		\captionsetup{type=figure}
		\hspace{-1em}
		\includegraphics[width=\textwidth]{figures/gnuplot/avg_cin_pre_count}
	\end{minipage}
	\captionof{figure}{Scatter plot showing the spread of average amount of call
sites contained in functions per file in the Benchmarks Suite. The files are
grouped by Benchmark along the horizontal axis.}
	\label{fig:benchmarks_avg_cin_lambda_per_file}
\end{centering}

The final function property we profile is the amount of call sites invoking a
function. As the first clause of our CNF form states, if the function is not
exported, and there's only one invocation of the function, it is an easy
decision to inline.

However, by graphing the spread of how many call sites each files' functions
have on average, it gives us an idea of how many more functions could
potentially be inlined if just some of its invocations are inlined.
Figure~\ref{fig:benchmarks_avg_scc_lambda_per_file} shows us a scatter plot of
this data across all the test files used in the Benchmark Suite. The few
functions in the bottom right corner of the plot are functions which have no
call sites inside the same file, but are exported, meaning that if the
benchmark's program files were statically linked during compilation, there would
be no functions with zero static call sites.

What the data in Figure~\ref{fig:benchmarks_avg_scc_lambda_per_file} tells us,
is that the majority of functions are called just once. Seeing as Jive has not
statically linked the files in the benchmark during the creation of the RVSDG,
this is understandable. If Jive was able to build the RVSDG with statically
linked files, our suspicion is that the range of the vertical axis, and spread data along it, would most likely grow.

\begin{centering}
	\noindent\begin{minipage}{\textwidth}
		\captionsetup{type=figure}
		\hspace{-1em}
		\includegraphics[width=\textwidth]{figures/gnuplot/avg_scc_pre_count}
	\end{minipage}
	\captionof{figure}{Scatter plot showing the spread of average Static Call
Counts for functions per file in the Benchmarks Suite. The files are grouped by
Benchmark along the horizontal axis.}
	\label{fig:benchmarks_avg_scc_lambda_per_file}
\end{centering}

However, as the data stands, we can see that if a function is not exported, and
two of its invocations are inlined, there is a decent chance of applying DCE on
the functions.

\subsection{Profiling SPEC2006's \applyNode s}
\label{sub:res:ic_profiling}

In our profiling, the data we are most interested relates to the \applyNode s in
the benchmarks. First we check in
Figure~\ref{fig:statically_known_applies_vs_all_applies} how many \applyNode s
each benchmark contains, and which of these we can potentially inline. Our
definition of \textit{statically known} apply-\textit{nodes}, is as follows: if
the $\lambda$-node invoked is directly connected to the \applyNode~in the RVSDG,
or resides within a $\phi$-region, which in turn is connected directly to the
\applyNode , it is statically known.

We cannot inline \applyNode s which are not statically known, and since Jive is
unable to link files at compile time, there is conceivably large portion of
\applyNode s not statically known. If Jive had been able to create and process
RVSDGs composed of linked object files, instead of just disparate files, the
proportion of statically known \applyNode s may have been higher.
Figure~\ref{fig:statically_known_applies_vs_all_applies} shows the (linear)
average count of \applyNode s per SPEC2006 Benchmark in the left bar, and the
(linear) average count of the statically known in the right bar.

\begin{centering}
	\noindent\begin{minipage}{\textwidth}
		\captionsetup{type=figure}
		\hspace{-1em}
		\includegraphics[width=\textwidth]{figures/gnuplot/avg_static_and_not_apply_nodes}
	\end{minipage}
	\captionof{figure}{Histogram showing the average count of \applyNode s (left
bar), and the average count of statically known \applyNode s (right bar), per
benchmark.}
	\label{fig:statically_known_applies_vs_all_applies}
\end{centering}

As Figure~\ref{fig:statically_known_applies_vs_all_applies} also shows, there
are only two benchmarks with more than ten statically known \applyNode s and
$\lambda$-nodes each. This time \textit{gcc} is the one with numbers high above
the rest with regards to the $\lambda$-node average, but what is shown is that
\textit{gcc}'s \applyNode~average also stands above the rest. Further
investigation reveals that there is one lone file inside the Benchmark which has
over six thousand \applyNode s and almost eight hundred $\lambda$-nodes. This
particular file contains almost only functions, many of which invoke multiple
others, thereby achieving these numbers.

\begin{centering}
	\noindent\begin{minipage}{\textwidth}
		\captionsetup{type=figure}
		\hspace{-1em}
		\includegraphics[width=\textwidth]{figures/gnuplot/avg_pc_cpc_pre_count}
	\end{minipage}
	\captionof{figure}{Scatter plot showing the spread of average Parameter Count
per file in the Benchmarks (plus signs), and the spread of average Constant
Parameter Count (depicted by tilted plus signs). The files are grouped by
Benchmark along the horizontal axis.}
	\label{fig:benchmarks_avg_pc_cpc}
\end{centering}

The data spread in Figure~\ref{fig:benchmarks_avg_pc_cpc} shows that there are
few function calls with up to six parameters, some of which have up to five
constant parameters in the invocation. The region of average parameter counts is
most dense in the area between $1.5$ and $2.5$ on the vertical axis. To avoid
inlining the majority of function calls, the lower limit of constant parameters
needed should be above two. Thus we avoid code size explosion by hindering
function invocations with only one or two constant parameters, while we permit
function invocations with more to become inlined.

Finally, one of the most important properties of a call site wrt. inlining is
the amount of nested loops it resides within.
Figure~\ref{fig:benchmarks_avg_lnd_per_file} plots the average LND count of the
call sites in a Benchmark Suite test file. The scatter plot in
Figure~\ref{fig:benchmarks_avg_lnd_per_file} tells us that there are no call
sites (\applyNode s) residing within more than two nested loops. While research
shows that most of a programs' execution time is spent inside of loops, it is
useful to have an idea of the proportion of call sites within one, two, or zero
loops. If the CNF clause checking for loop nesting depth of our CNFs final form
set the bar at three or higher, the clause would be useless as there are no such
call sites present in the Benchmark Suite's test files.

\begin{centering}
	\noindent\begin{minipage}{\textwidth}
		\captionsetup{type=figure}
		\hspace{-1em}
		\includegraphics[width=\textwidth]{figures/gnuplot/avg_lnd_pre_count}
	\end{minipage}
	\captionof{figure}{Scatter plot showing the spread of average Loop Nesting
Depth for call sites per file in the Benchmarks Suite. The files are grouped by
Benchmark along the horizontal axis.}
	\label{fig:benchmarks_avg_lnd_per_file}
\end{centering}

\subsection{Final CNF}
\label{sub:res:final_cnf}

With the observations made from the profiling data, and some testing to find the
optimum values wrt. to average node count size in the benchmarks, the following
values were used for our test results:

\begin{centering}
\lstinline!(EXP == false && SCC = 1) || NC < 25 || CIN < 3 ||! \\
\lstinline!(NC < 200 && CPC > 2) || (NC < 200 && LND > 0)! \\
\end{centering}

\subsection{Inlining results}
\label{sub:res:inlining}

Table summarizing node count difference between benchmarks pre- and post-inlining:
\begin{table}[h]
	\centering
	\caption{My caption}
	\label{my-label}
	\footnotesize
	\begin{tabular}{|c|l|l|l|l|l|l|l|l|l|l|l|l|l|}
		\hline
		{\bf \rotatebox{60}{benchmarks}} & \multicolumn{1}{l|}{{\it \rotatebox{90}{bzip2} }} & \multicolumn{1}{l|}{{\it \rotatebox{90}{gcc} }} & \multicolumn{1}{l|}{{\it \rotatebox{90}{gobmk} }} & \multicolumn{1}{l|}{{\it \rotatebox{90}{gromacs} }} & \multicolumn{1}{l|}{{\it \rotatebox{90}{hmmer} }} & \multicolumn{1}{l|}{{\it \rotatebox{90}{h264ref} }} & \multicolumn{1}{l|}{{\it \rotatebox{90}{lbm} }} & \multicolumn{1}{l|}{{\it \rotatebox{90}{libquantum} }} & \multicolumn{1}{l|}{{\it \rotatebox{90}{mcf} }} & \multicolumn{1}{l|}{{\it \rotatebox{90}{milc} }} & \multicolumn{1}{l|}{{\it \rotatebox{90}{perlbench} }} & \multicolumn{1}{l|}{{\it \rotatebox{90}{sjeng}}} & \multicolumn{1}{l|}{{\it \rotatebox{90}{sphinx3} }} \\ \hline
		{\bf Average Node Count Pre-Inlining}  &                             &                             &                             &                             &                             &                             &                             &                             &                             &                             &                             &                             &                             \\ \hline
		{\bf Average Node Count Post-Inlining} &                             &                             &                             &                             &                             &                             &                             &                             &                             &                             &                             &                             &                             \\ \hline
		{\bf \% Difference in Node Count}      &                             &                             &                             &                             &                             &                             &                             &                             &                             &                             &                             &                             &                             \\ \hline
		{\bf \# Files per Benchmark}           &                             &                             &                             &                             &                             &                             &                             &                             &                             &                             &                             &                             &                             \\ \hline
		{\bf \% of files total in Benchmark}   &                             &                             &                             &                             &                             &                             &                             &                             &                             &                             &                             &                             &                             \\ \hline
\end{tabular}
\end{table}

\todo[inline]{One graph showing the avg. wall clock ms time (per benchmark) on
y-axis, and the average amount of apply nodes (per benchmark) on x-axis.\\
Show that there is no strong correlation between the two.}

\begin{centering}
	\noindent\begin{minipage}{\textwidth}
		\captionsetup{type=figure}
		\hspace{-1em}
		\includegraphics[width=\textwidth]{figures/gnuplot/average_static_known_minus_loop_brkrs_post_pre}
	\end{minipage}
	\captionof{figure}{Scatter plot showing the sprad of the benchmarks' wall
clock duration (y-axis) with the average amount of apply nodes before inlining
(x-axis).}
\end{centering}

\todo[inline]{Check whether there is a strong correlation between the wall clock
time spent, vs Calls-In-Node combined with amount of lambdas/apply nodes.}
