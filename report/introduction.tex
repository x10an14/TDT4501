% !TEX root = ./report.tex

\section{Introduction}
\label{introduction}

Since the 1950s, compilers have played an important role in the way programming
code is translated into machine languages. In broad terms, compilers perform two
things; The translation of human-readable code to machine language, and
optimizing the translated programs. One such optimization is inlining.

\todo[inline]{Need inlining example here.  Plan to do it with code / latex
listing}

Inlining is a straight-forward optimization, which replaces the call of a
function with its body. Its benefits include removal of function call overhead
and unveiling of additional optimizations. The drawbacks are potential code
duplication, work duplication, as well as longer execution times for the
compilation of the program. Not all functions are straight-forward to inline,
recursive functions being part of this subset.

We will also discuss how the inlining heuristics used, and how these decisions
can be influenced/changed. Further details of the assignment of this paper can
be found in Appendix \ref{app:projdesc}.

\subsection{Report Outline}

\todo[inline]{Todo: Describe layout/outline of paper. What does each section in
turn discuss?}

This paper \todo{Todo: Fill in for the empty sections as they come.}details the
inliner for the new compiler backend, Jive, the decisions made for its
architecture, and evaluates its performance. Jive takes code in intermediate
representation (IR) as input and works on a new IR representation, the
regionalized value-state dependency graph (RVSDG\footnote{Detailed in Section
\ref{background:RVSDG}.}).
