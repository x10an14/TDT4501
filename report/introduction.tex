% !TEX root = ./report.tex

\section{Introduction}
\todo[inline]{Describe layout of paper. What does each section in turn discuss?}

\subsection{Problemsetting}

In these days, optimizing data movemement with regards to the memory hierarchy
in computers, is much more relevant than optimizations removing or simplifying
operations/computations. The removal of the CPU cycles spent waiting for the
necessary data to traverse to the top of the memory hierarchy is a much more
rewarding optimization, than optimizing away the instruction counts using the
on-chip ALUs in a processor.

This paper explores when inlining should be performed in the Jive compiler,
using the new novel approach of the Regionalized Value-State Dependency Graph
(\textbf{RVSDG})\footnote{See Appendix \ref{app:reportdesc} for a the
problemsetting in its original PDF.} to allow the compiler perform just such
optimizations on inlined code.

The RVSDG (Section \ref{background:RVSDG}) substitutes the known Control Flow
Graph (CFG), moving the focus of the program flow graph away from explicitly
describing operations and implicitly showing datamovements performed in the
program, and instead doing the opposite: explicitly describing the data
movement, and implicitly describing the operations performed by the program.
