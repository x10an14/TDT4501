% !TEX root = ./report.tex

\section{Introduction}
\todo[inline]{Describe layout of paper. What does each section in turn discuss?}

\subsection{Problemsetting}

\todo[inline]{Re-state the assignment: Inlining -> Easy, what is it, benefits/drawbacks. \\
``This paper details the problems and benefits of inlining'' \\
Introduce Jive in a few sentences, say that it's further introduced in section
2.}

Inlining is a simple and straight-forward technique used in code compilation,
where one replaces the call of a function with the body of said function. Its
benefits include removal of function call overhead, and more importantly,
unveiling of additional potential optimizations in the code. However, the
drawback is potentially increased code size, and longer execution times for the
compilation of the program.

This paper details the problem and benefits of inlining wrt. the Jive \todo{
insert reference} compiler. The Jive compiler is a new backend compiler using
intermediate representation (IR), and an alternative to the control flow graph
(CFG); the regionalized value-state dependency graph (RVSDG\footnote{Detailed in
Section \ref{background:RVSDG}.}).

Further details of this assignment can be found in Appendix
\ref{app:reportdesc}.
