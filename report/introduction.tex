% !TEX root = ./report.tex

\section{Introduction}
\todo[inline]{Describe layout of paper. What does each section in turn discuss?}

\subsection{Problemsetting}

\todo[inline]{Re-state the assignment: Inlining -> Easy, what is it, benefits/drawbacks. \\
``This paper details the problems and benefits of inlining'' \\
Introduce Jive in a few sentences, say that it's further introduced in section
2.}

Inlining is a simple and straight-forward technique used in code compilation,
where one replaces the call of a function with the body of said function. Its
benefits include removal of function call overhead and unveiling of additional
potential optimizations in the code. The drawbacks are potentially increased
code size as well as longer execution times for the compilation of the program.

The contribution of this paper is an inliner for the \todo{insert reference}Jive
compiler, and detailing the problems and benefits of inlining wrt. the Jive
compiler. Jive is a new backend compiler which works on intermediate
representation (IR) code, and optimizes it with a new type of graph; the
regionalized value-state dependency graph (RVSDG\footnote{Detailed in Section
\ref{background:jive}.}).

Further details of this assignment can be found in Appendix
\ref{app:projdesc}.
