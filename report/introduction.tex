% !TEX root = ./report.tex

\section{Introduction}
\todo[inline]{Describe layout of paper. What does each section in turn discuss?}

\subsection{Problemsetting}

\todo[inline]{Re-state the assignment: Inlining -> Easy, what is it, benefits/drawbacks. \\
``This paper details the problems and benefits of inlining'' \\
Introduce Jive in a few sentences, say that it's further introduced in section
2.}

Inlining is a straight-forward technique used in code compilation, which
replaces the call of a function with the body of said function. Its benefits
include removal of function call overhead (1) and unveiling of additional
potential optimizations in the code (2). The drawbacks are potentially increased
code size (3), as well as longer execution times for the compilation of the
program (4).

The contribution of this paper is an inliner for the Jive backend
compiler\footnote{Detailed in Section \ref{background:jive}.}, detailing the
problems and benefits of inlining wrt. inlining in the Jive compiler. Jive is a
new backend compiler which works on intermediate representation (IR) code, and
performs the typically expected compiler techniques and optimizations on said IR
code with the help of a new type of graph; the regionalized value-state
dependency graph (RVSDG\footnote{Detailed in Section \ref{background:RVSDG}.}).

Further details of this assignment can be found in Appendix
\ref{app:projdesc}.
