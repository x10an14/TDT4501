% !TEX root = ./report.tex

\section{Introduction}
\label{introduction}

Since the 1950s, compilers have been translating higher-level programming
languages into machine languages. The purpose of a compiler is two-fold:
translate human-readable code into machine language, and optimize the translated
code. Compilers use many code optimization techniques, in order to improve the
quality of the emitted code. Common techniques are \textit{Common Subexpression
Elimination} (CSE)~\cite[Ch. 8.5]{DragonBook}, \textit{Dead Code Elimination}
(DCE)~\cite[Ch. 8.5]{DragonBook}, and \textit{inlining}~\cite[Ch.
12.1]{DragonBook}.

Inlining replaces the call site of a function with its body.
Listing~\ref{lst:inlining} shows the definition of functions \lstinline!A()! and
\lstinline!B()!. If \lstinline!A()! is inlined into \lstinline!B()!, the body of
\lstinline!B()! becomes \lstinline!return y + 3 + 2!, permitting
\textit{Constant Folding} (CF)~\cite[Ch. 8.5]{DragonBook} to replace
\lstinline!3+2! with \lstinline!5!.

\begin{centering}
	\noindent\begin{minipage}{\textwidth}
		\begin{CenteredBox}
		\begin{lstlisting}[style=global_customcpp]
int A(int x){
	return x + 3;
}

int B(int y){
	return A(y) + 2;
}
		\end{lstlisting}
		\end{CenteredBox}
	\end{minipage}
	\captionof{lstlisting}{C/C++ code showing the definitions of \lstinline!A()!
and \lstinline!B()! when exemplifying inlining and CF.}
	\label{lst:inlining}
\end{centering}

The benefits of inlining are mainly two-fold: The first one is the removal of
function call overhead. Function call overhead is the cost in memory needed on
the stack, as well as the CPU cycles needed for setting up and performing the
call. The second is the potential for unveiling the application of additional
optimizations such as CF, demonstrated in Listing~\ref{lst:inlining}.

The drawbacks of inlining are code duplication, and an increased compile time.
In specific situations, work duplication can also occur~\cite{GHCPaper}.
Listing~\ref{lst:code-dup} exemplifies a situation where inlining can lead to
code duplication if either of \lstinline!C()!'s invocations in \lstinline!D()!
are inlined. This is due to the big expression \lstinline!e! being copied into
\lstinline!D()!. However, if both invocations are inlined, code code duplication
might be mitigated through CSE. Additionally, if both of \lstinline!C()!'s
invocations are inlined, then the function body of \lstinline!C()! may be
removed through the application of DCE since \lstinline!C()! is
\lstinline!static! and no longer invoked.

\begin{centering}
	\noindent\begin{minipage}{\textwidth}
		\begin{CenteredBox}
		\begin{lstlisting}[style=global_customcpp]
static int C(int a){
	return e; //Big expression, depending on a
}

int D(int x, int y){
	return C(x) + C(y);
}
		\end{lstlisting}
		\end{CenteredBox}
	\end{minipage}
	\captionof{lstlisting}{C/C++ code showing the definitions of
\lstinline!C()! and \lstinline!D()!, when exemplifying inlining and code
duplication.}
	\label{lst:code-dup}
\end{centering}

If inlining is performed blindly on all function call sites, non-termination of
the compilation can occur. This can happen when the compiler attempts to inline
recursive functions. Hence, recursive functions need to be handled carefully.
The predominant approaches for handling inlining of recursive functions are:

\begin{enumerate}

	\item Avoid non-termination of the compilation by only inlining recursive
functions to a certain depth~\cite{GHCPaper,InlineWhenHowSerrano}, and
therefore breaking the recursive cycle.

	\item In a mutually recursive environment, decide on one or more functions
to be \textit{loop breakers}, and mark them to never be inlined. Loop breakers
are chosen so that the recursive call cycle will be broken in the mutually
recursive environment. Having chosen correct loop breakers permits inlining of
the remaining recursive functions in the mutually recursive environment,
\textit{without} risking non-termination of the
compiler~\cite{BasMscThesis,GHCPaper}.

\end{enumerate}

This report describes the construction of an inliner for the Jive compiler
backend, detailing its design and architecture. Jive uses an
\textit{intermediate representation} (IR) called the \textit{Regionalized Value
State Dependence Graph} (RVSDG)~\cite{RVSDG:HiPEACpaper}.

The RVSDG described in Section~\ref{background:RVSDG},
is a \textit{demand-based} and \textit{directed acyclic graph} (DAG) where nodes
represent computations, and edges represent the dependencies between these
computations.

Section~\ref{sec:scheme} details the scheme of our project, and explains how the
inliner is able to handle recursive functions, as well as how the inliner
permits the configuration of different heuristics to allow rapid exploration of
the parameter space. How the RVSDG affects the design of an inliner, and the
algorithms used by the heuristics deciding what to inline, are also detailed in
this section.

In Section~\ref{sec:methodology}, we describe how the implemented inliner is
evaluated, as well as the inlining heuristics we chose for the SPEC2006
Benchmark Suites used for testing. Section~\ref{sec:res} reports on the
characteristics of the SPEC2006 Benchmark Suites used for the testing, as well
as the results of the inliner. Also in Section~\ref{sec:res}, we focus on how
different heuristics have different consequences, such as code duplication, and
others.

In Section~\ref{sec:related_work} we summarize the existing related literature
found in our background study for this project. Following, in
Section~\ref{sec:conclusion}, we conclude, before we finally discuss ideas for
potential further research in Section~\ref{sec:further_work}. A detailed
description of the project assignment can be found in
Appendix~\ref{app:projdesc}.

\todo[inline]{Figure out what to write about the below, and where in the report
to put it...}

Focus is put on whether the RVSDG simplifies or complicates the implementation
of the inliner, as well as the impact of the RVSDG on an inliner, and the
process of inlining, compared to commonly used IRs.
