\documentclass{article}
\usepackage[scale=0.75]{geometry}

\title{Project Description: An Inliner for the Jive compiler}
\author{Nico Reissmann}
\date{\today}

\begin{document}
\setlength{\parskip}{1ex}
\setlength{\parindent}{0pt}

\maketitle

Compilers have become an essential part of every modern computer system since their rise along with
the emergence of machine-independent languages at the end of the 1950s. From the start, they not only
had to translate between a high-level language and a specific architecture, but had to incorporate
optimizations in order to improve code quality and be a par with human-produced assembly code. One
such optimization performed by virtually every modern compiler is \textit{inlining}. In principle,
inlining is very simple: just replace a call to a function by an instance of its body. However, in
practice careless inlining can easily result in extensive \textit{work} and \textit{code
duplication}. An inliner must therefore decide carefully when and where to inline a function in
order to achieve good performance without unnecessary code bloat.

The overall goal of this project is to implement and evaluate an inliner for the Jive compiler
back-end. The project is split in a practical and an optional theoretical part. The practical part
includes the following:

\begin{itemize}
	\item Implementation of an inliner for the Jive compiler back-end. The inliner must be able to
	handle recursive functions and allow for the configuration of different heuristics to permit
	rapid exploration of the parameter space.
	\item An evaluation of the implemented inliner. A particular emphasis is given to different
	heuristics and their consequences for the resulting code in terms of work and code duplication.
\end{itemize}

The Jive compiler back-end uses a novel intermediate representation (IR) called the Regionalized
Value State Dependence Graph (RVSDG). If time permits, the theoretical part of the project is going
to clarify the consequences of using the RVSDG along with an inliner. It tries to answer the
following research questions:

\begin{itemize}
	\item What impact does the RVSDG have on the design of an inliner and the process of inlining?
	\item Does the RVSDG simplify/complicate the implementation of an inliner and the process
	of inlining compared to other commonly used IRs?
\end{itemize}

The outcome of this project is threefold:

\begin{enumerate}
	\item A working implementation of an inliner in the Jive compiler back-end fulfilling the
		aforementioned criteria.
	\item An evaluation of the implemented inliner.
	\item A project report following the structure of a research paper.
\end{enumerate}

\end{document}
