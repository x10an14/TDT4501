% !TEX root = ./report.tex

\clearpage
\section{Conclusion}
\label{sec:conclusion}

In this section we first \nolinebreak{re-iterate} our results from
Section~\ref{sec:res}, and reflect on the implications Jive's status has on our
results. Thereafter, we discuss the what we consider to be the more crucial
points of our inliner, before we finish with acknowledging all who helped with
this project.

\subsection{Summary \& reflection of results}
\label{sub:conclusion:summary}

The implemented inliner's effect on the linear average of operations in the test
files from the SPEC2006 Benchmark Suite is disputable, due to the fact that Jive
is still under development. This brings the risk that the compilations are not
\nolinebreak{bug-free}, and the challenge that it is not able to statically link
files at compile time. Nor is Jive able at this point in time to produce
executable code we could test.

Yet in spite of these drawbacks, Jive shows promise, enabling our inliner to
reduce the linear average of operations across the Benchmark Suite with $6.78\%$
and $6.98\%$, when traversing the call sites in a \nolinebreak{top-down} or
bottom up order respectively. If improvements like the ones discussed in
Section~\ref{sec:further_work} could be implemented, we have little doubt that
the strengths exhibited by our inliner could be further improved and utilized to
beneficial effect.

Our scientific contribution in this project is the notion of traversing the call
sites in a \nolinebreak{top-down} or \nolinebreak{bottom-up} order in the RVSDG
of the program. While Section~\ref{sub:fw:call_site_visit_order} expands more on
our thoughts on this, we do want to point out that our implemented traverser
described in Section~\ref{sec:scheme} has an algorithmic complexity of $O(N^2
Log N)$. Effort was not put into optimizing the traversal algorithm, since that
was not a part of the project description. Hence, we speculate that while the
traversal ordering could be worthwhile, some effort should be put into optimizing the traversal of the algorithm.

\subsection{Acknowledgments}

First of all, I would like to thank Nico Rei\ss mann for his tutelage, guidance,
and last but not least, patience in his role as my supervisor for this project.
While the experience has taught me a great deal of things, his contribution of
efforts and his own time to both the project and my supervision cannot be
understated.

While this project is not on par with a Master Thesis, it is the most
challenging one I've had to date. Thus, I'd also like to thank Torje Digernes,
Bj\o rn \AA ge Tungesvik, Einar Johan S\o m\aa en, J\o rgen Kvalsvik, for their
tips, debugging help, support, and tutorials teaching me to utilize the tools
needed for this project, some of which I had little prior experience with.

Finally I'd also like to thank Dag Frode Solberg, and all the others who helped
me by reading through my reporting, and/or just giving me their time and effort
in improving my project and report.
