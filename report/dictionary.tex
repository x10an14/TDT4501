% !TEX root = ./report.tex

\section{Dictionary}

\todo[inline]{Figure out the terms found in papers...}

\subsection{\cite{GHC-paper}}
\begin{itemize}

	\item Section 1
\begin{itemize}

	\item ``inlining subsumes''
Something gets removed by inlining.

	\item ``lexical scopes''
Scope of a program/function/block/variable.
From StackOverflow:
\url{http://stackoverflow.com/questions/1047454/what-is-lexical-scope}


	\item ``pure'' (\textit{language})
Program must always return same results with same inputs. There can be no
``side-effect'' which changes the state of the program.

	\item ``explicitly typed'' (\textit{language})
That the compiler knows at compile time (static) what types all the variables
and return values have.
Great explanation: \url{http://programmers.stackexchange.com/questions/181154/type-systems-nominal-vs-structural-explicit-vs-implicit}

	\item ``strictness analysis''
\todo[inline]{Need more research}
\url{http://en.wikipedia.org/wiki/Strictness_analysis}

	\item ``\textit{let}-floating'' (\textit{Haskell})
Nico: Not important

	\item ``name capture''
Making sure each name is an unique identifier. This gets complicated when
compiler has to rename things for program correctness. Like when inlining.

\end{itemize}

	\item Section 2
\begin{itemize}

	\item ``\textit{$\beta$}-reduction''
	\item ``invariant'' (\textit{language artifact/variable/expression?})
	\item ``\textit{trivial-constructor-argument invariant}'' (\textit{Haskell?
})
	\item ``divergent computations''

	\item ``closure'' (\textit{scopes of functions?})
Ensuring correct lexical scope for (free)variables, mainly needed in languages
permitting nested function.

	\item ``lambda calculus''
	\item ``literals''
The string ``abcd'' is a literal. As is the number 1, when used to assign value
to an integer, just like the string is used to assign value to a string-variable.

	\item ``primitive operators''
Operators that are in one way ``basic'', but not necessarily only (nor all) of
the operations supported by hardware. Think of it as the most basic operators
which together build up all others.

\end{itemize}

	\item Section 3
\begin{itemize}

	\item ``bound variable''
A variable that is not free, that is, a variable which has been specified as an
input to a function call. (The opposite of a free variable, which is a variable
used in a function, but declared outside of it, and not listed as one of its
parameters).

	\item ``recursive binding groups''
	\item ``strongly-connected components''
	\item ``one-shot lambdas''
	\item ``contravariantly''
	(\textit{(..) it appears contravariantly in its own definition.})
	\item ``ùntyped programs''
	\item ``pathological programs''
	\item ``static analysis''

\end{itemize}

	\item Section 4
\begin{itemize}

	\item ``hash-consing''

\end{itemize}

\end{itemize}
